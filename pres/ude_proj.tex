\documentclass[10pt,pdf,hyperref={unicode}]{beamer}

% \documentclass[aspectratio=43]{beamer}
% \documentclass[aspectratio=1610]{beamer}
% \documentclass[aspectratio=169]{beamer}

\usepackage{lmodern}
\usepackage[russian]{babel}

% подключаем кириллицу 
%\usepackage[T2A]{fontenc}
\usepackage[T1]{fontenc}
\usepackage[utf8]{inputenc}
\usepackage{bm}

% отключить клавиши навигации
\setbeamertemplate{navigation symbols}{}

% тема оформления
\usetheme{Madrid}

% цветовая схема
\usecolortheme{whale}

\title[UDE]{Универсальные дифференциальные уравнения}   
\subtitle{Комбинация лучших сторон разных подходов.}
\author{Влад Темкин} 



\institute[HSE] % (optional)
{
	Высшая Школа Экономики\\
	Факультет Физики
}

\date[\today]
{Стохастические процессы и моделирование}

\AtBeginSection[]
{
	\begin{frame}
		\frametitle{План доклада}
		\framesubtitle{Основные моменты}
		\tableofcontents[currentsection]
	\end{frame}
}

\begin{document}
	
	\begin{frame}
		\titlepage
	\end{frame} 
	
	
	\begin{frame}
		\frametitle{План доклада} 
		\framesubtitle{Основные моменты}
		\tableofcontents[pausesections]
	\end{frame}


	\section{Подходы для описания мира}
	
		\begin{frame}
			\frametitle{Подходы для описания мира} 
				всем привет
		\end{frame}
	
	
		\subsection{Физические модели}
		
			\begin{frame}
				\frametitle{Подходы для описания мира} 
				\framesubtitle{Физические модели}
					\begin{columns}
						\column{0.5\linewidth}
						\begin{center}
							\begin{itemize}
								\item
								\item
								\item
							\end{itemize}
						\end{center}
						\column{0.5\linewidth}
						я начал презу
					\end{columns}
			\end{frame}
		
		
		\subsection{Большие данные}
		
			\begin{frame}
				\frametitle{Подходы для описания мира} 
				\framesubtitle{Большие данные}
					\begin{columns}
						\column{0.5\linewidth}
						второй слайд
						\column{0.5\linewidth}
						\begin{center}
							\begin{itemize}
								\item
								\item
								\item
							\end{itemize}
						\end{center}
					\end{columns}
			\end{frame}
			
			
	\section{Комбинирование двух подходов}
		
		\begin{frame}
			\frametitle{Комбинирование двух подходов} 
			\framesubtitle{Различные практики}
				тут про нейронные диффуры 
		\end{frame}
		
		
	\section{Универсальные дифференциальные уравнения}
	
		\begin{frame}
			\frametitle{Универсальные дифференциальные уравнения} 
			\framesubtitle{Выделенный случай}
				диффур, заменяем слагаемое на нейронку 
		\end{frame}
	
	
	\section{Вспомогательные инструменты}
	
		\subsection{алгортим SINDy}
		
			\begin{frame}
				\frametitle{Вспомогательные инструменты} 
				\framesubtitle{Алгоритм SINDy}
					тут пару слов о том как робит синди  
			\end{frame}
		
		
	\section{Примеры применения UDE}
	
		\begin{frame}
			\frametitle{Примеры применения UDE} 
				задачи бывают всякие разные  
		\end{frame}
	
		
		\subsection{Модифицированная модель SEIR}
		
			\begin{frame}
				\frametitle{Примеры применения UDE} 
				\framesubtitle{Модифицированная модель SEIR}
					история о том как я код пиздил  
			\end{frame}
			
			
		\subsection{Модель Лотки-Вольтерра}
			
			\begin{frame}
				\frametitle{Примеры применения UDE} 
				\framesubtitle{Модель Лотки-Вольтерра}
					история о том как я код пиздил  во второй раз
			\end{frame}
		
		
	\section*{Литература}
	
		\begin{frame}
			\frametitle{Список литературы} 
				\begin{itemize}
					\item
				\end{itemize}
			
		\end{frame}

\end{document}